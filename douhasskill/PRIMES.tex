Latex document for development of the
unoptimized prime finder in haskell.

Let the bound be denoted B, the bound is the range for computing primes.
No primes can be even, as all even numbers are divisible by 2, all primes
must be odd. Therefore the domain of possible primes is denotable using
set builder notation as such;

$[x+2\mid x\leftarrow[1...B]]$, this is the domain of POSSIBLE primes.

The unoptimized prime finder does not take advantage of statistical methods
or parallel computation. Hence the form of the code will be small, sweet
and simple. When god speaks to us he does so with a lisp, like a nurse does
to a sick child.

let $[\tau]=[x+2\mid x\leftarrow[1...B]]$, multiples of a factor are divisible by that factor.
Hence no primes can be a multiple of a member of the domain TAU.

let $\lambda^m(\tau_i,B)= [\tau_i x\mid x\leftarrow[2..\dfrac{B}{2}]]$. This function takes a factor and returns a list containing
multiples of that factor up to the bound B/2. This serves as the basis of elimination. As no numbers
in these lists can be prime, as well as a cross elimination between lists as multiples can be contained
in multiple lists..... common multiples, between factors...... so the intermidate step between yielding primes
from these lists is preforming a cross elimination of the common multiples, which will significantly reduce the
computation time. At this point the primes are like blank spaces. The multiples are ordered in increasing order,
hence between each pair of multiples is a boundary where a prime IS present. For example....

[3,5,7] PRIMES
[6,9] ->  [7,8].
[5,10] -> [6,7,8,9].
[14,21] ->[15,16,17,18,19,20]. A key observation can be made, the universe between consecutive pairs of
multiples increases as the parent factor increases, or even as the size of the numbers increase with a constant
gap it results in increased computation load as larger numbers require longer compute times.
the gap between $\lambda^{m,\tau_i}_{i-1}\land\lambda^{m,\tau_i}_i$ increases, the universe refers to the distance between maximum and minimum points.

the problem is that even numbers are contained within the set of possible primes. but with haskell the even numbers
can be easily be removed using a bounded range, where the bounds are the min and max points of the multiple list, with
a step of 2, only even numbers are in the list, hence by applying a negation of this set, against the set prior.
primes are yielded within that boundary. $\color{wildstrawberry}\text{there may be a smarter way to do this.}$

$[\lambda^{m,\tau_i}_{i-1}...\lambda^{m,\tau_i}_{i}]=[\sigma^{i,i-1}]$, the set of possible primes. $[x\mid x\leftarrow[\lambda^{m,\tau_i}_{i-1}...\lambda^{m,\tau_i}_i]]\{x\mod{2}=[\tau^e]$
the set containing even numbers within the bound.
$\lambda^\neg(a,b) = \neg a \{b\in a$. simple negation logic
$\lambda^\neg([\sigma^{i,i-1}],[\tau^e])=\neg [\tau^e]\in [\sigma^{i,i-1}]$. application of prior logic to remove evens from bounded range.
the key is the negation, now working backwards it should be possible to construct a large equation encapsulating
the entire mechanisim. as such;


as $[\lambda^m(x,y)] = [yx\mid y\leftarrow[x...\dfrac{b}{2}]]$ and $[\tau] = [x+2\mid x\leftarrow[1...b]]$.
by using set builder notation it is easy to apply the lambda multiple logic to the
factor set. $[\lambda^m(x+2)\mid x\leftarrow [1...b]]=[[\tau^{\lambda_m}]]$, as
$[[\lambda^m(\tau)]]=[yx\mid y\leftarrow [a...\dfrac{b}{2}]\ x\leftarrow[a+2\mid a\leftarrow[1...b]]]$
$\tau$
so of form;  $\begin{bmatrix}
a&b\\
c&d
\end{bmatrix}$, this can be understood as something like a net. the multiples and the parent factor form the threads, concatentation of
threads form the net. as threads can share common factors, a reduction can be applied to have the net only contain unique values, this results
in the threads taking a non-uniform nature and hence the "net" object becomes sparse.
similar negation logic can be applied as is done in the negation of evens from the bounded multiples.
this should yield a sparse net, from a full net. the following logic attempts to roughly describe this;
1) full net, $[[\lambda^m(\tau)]]=[yx \mid y\leftarrow[a...\dfrac{b}{2}]\ x\leftarrow[\tau]]$
2) negation function,  $[\lambda^\neg(a,b)] = \neg b\{a_{i+n} \in b \mid i,n \in[a_i...a_{i+n}]$
the cross negation of the full net will require the application of the negation logic to a 2-tuple of lists.
hence a two dimensional field.
3) cross negation of full net.
3a) indexing the net.

$\begin{bmatrix} \lambda^m_\tau&...&\lambda^m_{\tau+b}\\.&...&.\\.&...&.\\
\lambda^{m+\tiny\frac{b}{2}}_\tau&...&\lambda^{m+\tiny\frac{b}{2}}_{\tau+b}\end{bmatrix}$, INDEXING METHOD OF THE NET, ALINING M and TAU with its y and x internal definitions.










$\color{WildStrawberry}\text{MAIN SECTION.}$ ! above is just rough draft and thoughts.

