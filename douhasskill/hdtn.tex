% preamble.tex
\documentclass{article}
\usepackage{amssymb}
\usepackage{amsmath}
\usepackage{color}


% main.tex
\title{Hello World!}
\author{Jeevman Banggore}


\begin{document}
\maketitle
\newpage
\section{HDTN:Introduction}
A highly divisible triangle number is the sum of natural numbers.\\
The question posed in the archive of project euler, asks, what is the value of the first triangle number to have over 500 divisors.
This seems slightly related to prime numbers as a prime is a number divisible by one and itself, so then no HDTN should be prime.
The only prime contained in the HDTN sequence is 3 as the sum of 1 + 2. All numbers thereafter appear to be divisible.
The point of these documents is to develop a sense of mathematical understanding in conjunction with haskell inorder to provide
solutions that are as near to optimal as possible. This includes some deductive reasoning backed with inextensive proofs. 

\section{HDTN:Disstillation}
This section contains considered insights from the sequence of HDTN, the next section will be forming general assumptions 
that can be made based on these insights.\\
{\color{white}-}\\
Consider;\ [1,3,6,10,15], HDTN sequence of cardinality 5.\\
0+1 = 1\\
1+2 = 3\\
1+2+3 = 6\\
1+2+3+4 = 10\\
1+2+3+4+5 = 15\\
Algebraic mapping of such;\\
$0+1=1\\a+b=b\ when\ a=0$\ hence\\ $1+2+3+4+5=15\\a+b+c+d+e=f$\\ with a further simplification as such;\\
$x_1+x_2+x_3+x_4+x_5=y_5$\\
{\color{white}-}\\
induction into sets;\\
$x_i\in[x_n...x_d]$\ $y_i\in[y_n...y_d]$\\ as a set X maps to a singular value, it is expressed as such;\\
$f([x_i...x_d])=y_d$, where the function is of the form;
\begin{equation}f([x_i...x_d])=\sum_i^d=x_i+x_{i+1}\end{equation}
\newpage
\section{HDTN:Considerations A}
Carefully. Very, carefully, avoid a pitfall, that being the dissolution of a pattern with hidden infinities.
The above function, is hard, I think there are very few methods that yield insights into the function and 
generation of the final HDTN. Taking the rate of change may be the most fruitful method. First I will explain
what is ment by a pitfall. Quite simply. It is a hidden polynomial. Where the degree of the polynomil i
dependant on the cardinality of the domain.\\ Writing the full form of the above function with cardinality
4 results in such;\\ $f([x_1...x_4])=x_1+x_2+x_3+x_4$ if not clear at this point, let $x = ax$ then;\\
$f([x_1...x_4])=a_1x_1+a_2x_2+a_3x_3+a_4x_4$, being a polynomial of the Nth degree where N = $\mid[x]\mid$\ \ : )\\
{\color{white}-}\\
I think, an easy way to do this is brute force approximation of the coefficients by using a boolean
distance function that operates over iter1 and iter2 if the loss is less at iter2 than at iter1 then 
the change in parameters is positive and approching approximation. This, becomes a difficult approach for larger
mapping functions as the amount of "false" positives increase. Keenly take note, this is therefore not a smart
approach. Unless guided by a quantum mechanisim.\\


\end{document}
