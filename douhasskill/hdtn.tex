% preamble.tex
\documentclass{article}
\usepackage{amssymb}
\usepackage{amsmath}
\usepackage{xcolor}


% main.tex
\title{Hello World!}
\author{Jeevman Banggore}


\begin{document}
\maketitle
\newpage
\section{HDTN:Introduction}
A highly divisible triangle number is the sum of natural numbers.\\
The question posed in the archive of project euler, asks, what is the value of the first triangle number to have over 500 divisors.
This seems slightly related to prime numbers as a prime is a number divisible by one and itself, so then no HDTN should be prime.
The only prime contained in the HDTN sequence is 3 as the sum of 1 + 2. All numbers thereafter appear to be divisible.
The point of these documents is to develop a sense of mathematical understanding in conjunction with haskell inorder to provide
solutions that are as near to optimal as possible. This includes some deductive reasoning backed with inextensive proofs. 

\section{HDTN:Disstillation}
This section contains considered insights from the sequence of HDTN, the next section will be forming general assumptions 
that can be made based on these insights.\\
{\color{white}-}\\
Consider;\ [1,3,6,10,15], HDTN sequence of cardinality 5.\\
0+1 = 1\\
1+2 = 3\\
1+2+3 = 6\\
1+2+3+4 = 10\\
1+2+3+4+5 = 15\\
Algebraic mapping of such;\\
$0+1=1\\a+b=b\ when\ a=0$\ hence\\ $1+2+3+4+5=15\\a+b+c+d+e=f$\\ with a further simplification as such;\\
$x_1+x_2+x_3+x_4+x_5=y_5$\\
{\color{white}-}\\
induction into sets;\\
$x_i\in[x_n...x_d]$\ $y_i\in[y_n...y_d]$\\ as a set X maps to a singular value, it is expressed as such;\\
$f([x_i...x_d])=y_d$, where the function is of the form;
\begin{equation}f([x_i...x_d])=\sum_i^d=x_i+x_{i+1}\end{equation}
\newpage
\section{HDTN:Considerations A}
Carefully. Very, carefully, avoid a pitfall, that being the dissolution of a pattern with hidden infinities.
The above function, is hard, I think there are very few methods that yield insights into the function and generation
of the final HDTN. Taking the rate of change may be the most fruitful method. First I will explain what is ment by a pitfall.
Quite simply. It is a hidden polynomial. Where the degree of the polynomial is dependant on the cardinality of the domain.\\
Writing the full form of the above function with cardinality 4 results in such;\\
$f([x_1...x_4])=x_1+x_2+x_3+x_4$ if not clear at this point, let $x = ax$ then;\\
$f([x_1...x_4])=a_1x_1+a_2x_2+a_3x_3+a_4x_4$, being a polynomial of the Nth degree where N = $\mid[x]\mid$\ \ : )\\
{\color{white}-}\\
Therefore I will make the assumption that being able to find the HDTN requires computing the roots of the polynomial via some method
and this, might be a more efficient way of computing a HDTN without brute force addition of every single natural value. I will also
make the second assumption that producing the divisible factors might come from deducing the coefficient of each respective term.\\
{\color{white}-}\\
Finding the roots of a polynomial is a classic ml problem, and as long as a steady pattern is present, then error correction with
a linear approach should suffice without any probablisitic mechanisims. I will also attempt to devise a hard formulation but I feel
my limitations as I think the formula will change dependant on the degree of the polynomial. Hence it will require devising a
function that alters a formula, adding and maybe removing terms to match the degree. Specifically what I warn against trying at the 
start of this section, as it involves conquring an infinity, and hence why ml is THE IDEAL approach. As it automates the traversal.
\\
First a clear decleration, that the $x_q$ terms do not need to be "found" as they are known, hence only $a_q$ coefficients.
$q \in [[i...n]\in[x_i...x_n]]$. Following is a small numerical example, but another decleration must be made in preparation. as ax = x.
and x is a natural number, the ax respective of a must result in an x that is a whole natural number.\\
An unavoidability is the presence of primes. Carefully taking note of the HDTN generation sequence, it includes primes and non-primes,
as it is the sequence of natural numbers. It can be declared that a HDTN is a highly related to a ratio between prime and non prime numbers.

\end{document}
