% preamble.tex
\documentclass{article}
\usepackage{amssymb}
\usepackage{amsmath}
\usepackage{color}


% main.tex
\title{Hello World!}
\author{Jeevman Banggore}


\begin{document}
\maketitle
\newpage
\section{HDTN:Introduction}
This is a LaTeX document backed with Haskell, displaying a brute
force method of computing the divisors of highly divisible triangle
numbers. Two methods will be used, initally brute force search,
and second a mathematical distillation of the sequence to optimize
the required computations towards a minimal point.
\section{Brute Force Search}
The following section describes the process of searching for divisors for HDTNs with a brute force approach. 
Through constructive mathematics a small set of functions can be denoted for finding HDTN, and the structure
should be similar to the unoptimized prime finder as essentially it is a prime check for non prime numbers.
But the key difference is calculating the divisors is essential as the divisors are being seached for,
not for numbers that lack divisors.
\subsection{BFS: General Intuitions}
1+2 = 3\\
1+2+3= 6\\
1+2+3+4 = 10\\
1+2+3+4+5 = 15\\
1+2+3+4+5+6 = 21\\
1+2+3+4+5+6+7 = 28\\
{\color{white}-}\\
A HDTN is the sum of consecutive natural numbers, not all natural numbers in the summing sequence is a
divisor of the HDTN. The minimum multiple is two as a double of x. Therefore all possible divisors lie
withing y/2. Looking at the above sequence, the difference between the last number in the summing sequence
and the sum increases by an exponential degree. For now a bound y/2 is sufficient, but an optimized search
would vary the bound dependant on y, possible allowing it to be y/4 and even lower, but implimenting such a
bound would require some precision as if the variable bound is incorrect then divisors will be missed during
the search.\\
1) HDTN lie in either the odd or even domain. Encapsulating both sides. Due to symetry being present in
both odd and even domains.\\
2) If a HDTN is odd it cannot have even divisors.\\
3) However a even HDTN can have odd divisors.\\
4) Hence even HDTN have more divisors as they lie in both the odd and even domain.\\
5) The set of possible divisors is from 2 to B where B is the bound.\\
6) If the HDTN is odd then the set of possible divisors only contains odd numbers.\\
7) If the HDTN is even then the set of possible divisors contain all natural numbers from 2 to B.
\subsection{BFS: Distillation of Intuitions}
First a classification function will be definied.
\begin{equation}
\sigma^A(x)= y 
\begin{cases}
1\mid (x\mod 2 = 0) \\
0\mid (x\mod 2 \not= 0)
\end{cases}
\end{equation}
Hence a HDTN can be classified as odd or even. Based on this classification two different search methods are
used. If odd, only odd factors are generated from 3 to the factor where y/3 = b. If even, all factors are
checked. From two, to the factor where y*2 = b.\\
The bounds can denoted as such;
\begin{subequations}
\begin{align}
	B^O\land B^E\\
	B^O(y) = y/3\\
	B^E(y) = y/2
\end{align}
\end{subequations}
\begin{subequations}
\begin{align}
	[F_{B^O}] = [F\in[3...\frac{y}{3}]\mid F]\\
	[F_{B^E}] = [F\in[3...\frac{y}{2}]\mid F]\\
	[F_{B^O}]_{\mid F_{B^O}\mid}*3=y\\
	[F_{B^E}]_{\mid F_{B^E}\mid}*2=y 
\end{align}
\end{subequations}
hence the maximal factor is computed, it is then infered that the respective set infers all divisors.
constructing a set of ones equal to the size of the respective set yields a state that when summed is 
equal to the amount of divisors respective of the HDTN. Simple, ez.\\
Written in full form it follows as such;
\section{BFS: Formalization}
\begin{subequations}
\begin{align}
	\mathbb{B}(x)= y\begin{cases}1\mid x\mod2=0\\
	0\mid x\mid2\not=0 \end{cases}\\
	[F({\mathbb{B}(x)})]=[a...b]
	\begin{cases}
		a=2,b=\dfrac{x}{2}\mid \mathbb{B}(x)=0\\
		{\color{white}-}\\
		a=3,b=\dfrac{x}{3}\mid\mathbb{B}(x)\not=0
	\end{cases}\\
	\sum_{i=0}^{\mid[F(\mathbb{B}(x))]\mid}=[F(\mathbb{B}(x))]_i
\end{align}
\end{subequations}
The above has been written into a Haskell program to show proof of concept and function.
\end{document}
