% preamble.tex
\documentclass{article}
\usepackage{amssymb}
\usepackage{amsmath}
\usepackage{xcolor}


% main.tex
\title{Hello World!}
\author{Jeevman Banggore}


\begin{document}
\maketitle
\newpage
\section{HDTN:Introduction}
A highly divisible triangle number is the sum of natural numbers.\\
The question posed in the archive of project euler, asks, what is the value of the first triangle number to have over 500 divisors.
This seems slightly related to prime numbers as a prime is a number divisible by one and itself, so then no HDTN should be prime.
The only prime contained in the HDTN sequence is 3 as the sum of 1 + 2. All numbers thereafter appear to be divisible.
The point of these documents is to develop a sense of mathematical understanding in conjunction with haskell inorder to provide
solutions that are as near to optimal as possible. This includes some deductive reasoning backed with inextensive proofs. 

\section{HDTN:Insights}
This section contains considered insights from the sequence of HDTN, the next section will be forming the general shape of
assumptions that can be made based on these insights.\\
{\color{white}-}\\
Consider;\ [1,3,6,10,15], HDTN sequence of cardinality 5.\\
0+1 = 1\\
1+2 = 3\\
1+2+3 = 6\\
1+2+3+4 = 10\\
1+2+3+4+5 = 15
\begin{equation}f^\prime(y,x) = \dfrac{y_n - y_\emptyset}{x_n,x_\emptyset}\end{equation} 
Simplistic form of rate of change with respect to another variable.\\
$consider; \\ 0+1=1\\a+b=b\ when\ a=0$\ hence\\ $1+2+3+4+5=15\\a+b+c+d+e=f$\ with further simplifications as such;\\
$x_1+x_2+x_3+x_4+x_5=y_5$








\end{document}
