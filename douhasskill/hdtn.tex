% preamble.tex
\documentclass{article}
\usepackage{amssymb}
\usepackage{amsmath}
\usepackage{color}


% main.tex
\title{Hello World!}
\author{Jeevman Banggore}


\begin{document}
\maketitle
\newpage
\section{HDTN:Introduction}
This is a LaTeX document backed with Haskell, displaying a brute
force method of computing the divisors of highly divisible triangle
numbers. Two methods will be used, initally brute force search,
and second a mathematical distillation of the sequence to optimize
the required computations towards a minimal point.
\section{Brute Force Search}
The following section describes the process of searching for divisors for HDTNs with a brute force approach. 
Through constructive mathematics a small set of functions can be denoted for finding HDTN, and the structure
should be similar to the unoptimized prime finder as essentially it is a prime check for non prime numbers.
But the key difference is calculating the divisors is essential as the divisors are being seached for,
not for numbers that lack divisors.
\subsection{BFS: General Intuitions}
1+2 = 3\\
1+2+3= 6\\
1+2+3+4 = 10\\
1+2+3+4+5 = 15\\
1+2+3+4+5+6 = 21\\
1+2+3+4+5+6+7 = 28\\
{\color{white}-}\\
A HDTN is the sum of consecutive natural numbers, not all natural numbers in the summing sequence is a
divisor of the HDTN. The minimum multiple is two as a double of x. Therefore all possible divisors lie
withing y/2. Looking at the above sequence, the difference between the last number in the summing sequence
and the sum increases by an exponential degree. For now a bound y/2 is sufficient, but an optimized search
would vary the bound dependant on y, possible allowing it to be y/4 and even lower, but implimenting such a
bound would require some precision as if the variable bound is incorrect then divisors will be missed during
the search.
\end{document}
