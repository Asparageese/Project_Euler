% preamble.tex
\documentclass{article}
\usepackage{amssymb}
\usepackage{amsmath}
\usepackage{color}

% main.tex
\title{Hello World!}
\author{Jeevman Banggore}


\begin{document}
\maketitle
\newpage
\section{HDTN:Introduction}
A highly divisible triangle number is the sum of consecutive natural numbers. The question being
proposed is how to find the divisors of hdtn, as a hdtn can either be odd or even, we assume
that there are two classifications of hypersymetry due to the existance of common multiples 
between factors. From those factors the primes must be accounted for as unique factors.
I think one can therefore disect and recombine some prime yielding logic to work for yielding
divisors, hence I will attempt to disect the prime finding code and alter its form to yield,
the divisors of a number, as a prime involves checking for possible divisors, and the question
is how to find the divisors of a hdtn, which might contain primes, non-primes and both primes and 
non-primes. : )\\
Following I will describe the structure of the document, Encapsulation, Consideration, Decomposition,
Recomposition.
\section{Encapsulation}
\subsubsection{1 consideration}
The question is how to find the divisors of hdtn, a hdtn can be either odd or even. The above is a function
from the PRIME finder document, allowing for classification of x as either odd or even. Naturally a question
arises, if hdtns can be odd and even, and no prime is even hence all primes lie in the odd domain, is it
possible that a hdtn of odd denomination is a prime, and hence has no divisors. The hdtn sequence is as
follows:\\(1+2)=3\\(1+2+3)=6\\(1+2+3+4)=10\\(1+2+3+4+5)=15\\(1+2+3+4+5+6)=21\\ I will limit my assumptions
to the above, assumption one as a prime number is a unique factor and hence a "building block" of other
numbers, it fills in the gaps left by multiples due to the fact that multiplication paths of factors cover
finite and cyclical points. Hence if the starting factors are two and three, gaps are left in 5, and 7 and 
so on. So primes are the factors that fill the implied missing spaces. Therefore a prime cannot be the sum of
consecutive natural numbers, starting from 1. As a small proof of this I will make use of two programs,
one, a prime number finder, and two a hdtn sequence yielder. Then it is implied that checking numbers from
the sets yielded by the prime number finder and hdtn sequence yielder should hold no duplicates, each number
set should be unique, and if such is true then no prime number is a hdtn, and therefore special rules may
be able to be derived for bounds based on the location of primes.
\subsubsection{Encapsulation: Odd and Even identity}
\begin{equation}
	f_{\not O\not E}(x) = y
	\begin{cases} True  \mid x \mod 2 = 0\\
	False \mid otherwise\end{cases}
\end{equation}

\end{document}
